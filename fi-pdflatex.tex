%%%%%%%%%%%%%%%%%%%%%%%%%%%%%%%%%%%%%%%%%%%%%%%%%%%%%%%%%%%%%%%%%%%%
%% I, the copyright holder of this work, release this work into the
%% public domain. This applies worldwide. In some countries this may
%% not be legally possible; if so: I grant anyone the right to use
%% this work for any purpose, without any conditions, unless such
%% conditions are required by law.
%%%%%%%%%%%%%%%%%%%%%%%%%%%%%%%%%%%%%%%%%%%%%%%%%%%%%%%%%%%%%%%%%%%%

\documentclass[
  digital,     %% The `digital` option enables the default options for the
               %% digital version of a document. Replace with `printed`
               %% to enable the default options for the printed version
               %% of a document.
%%  color,       %% Uncomment these lines (by removing the %% at the
%%               %% beginning) to use color in the printed version of your
%%               %% document
  oneside,     %% The `oneside` option enables one-sided typesetting,
               %% which is preferred if you are only going to submit a
               %% digital version of your thesis. Replace with `twoside`
               %% for double-sided typesetting if you are planning to
               %% also print your thesis. For double-sided typesetting,
               %% use at least 120 g/m² paper to prevent show-through.
  nosansbold,  %% The `nosansbold` option prevents the use of the
               %% sans-serif type face for bold text. Replace with
               %% `sansbold` to use sans-serif type face for bold text.
  nocolorbold, %% The `nocolorbold` option disables the usage of the
               %% blue color for bold text, instead using black. Replace
               %% with `colorbold` to use blue for bold text.
  lof,         %% The `lof` option prints the List of Figures. Replace
               %% with `nolof` to hide the List of Figures.
  lot,         %% The `lot` option prints the List of Tables. Replace
               %% with `nolot` to hide the List of Tables.
]{fithesis4}
%% The following section sets up the locales used in the thesis.
\usepackage[resetfonts]{cmap} %% We need to load the T2A font encoding
\usepackage[T1,T2A]{fontenc}  %% to use the Cyrillic fonts with Russian texts.
\usepackage[
  main=czech, %% By using `czech` or `slovak` as the main locale
                %% instead of `english`, you can typeset the thesis
                %% in either Czech or Slovak, respectively.
  english, german, russian, czech, slovak %% The additional keys allow
]{babel}        %% foreign texts to be typeset as follows:
%%
%%   \begin{otherlanguage}{german}  ... \end{otherlanguage}
%%   \begin{otherlanguage}{russian} ... \end{otherlanguage}
%%   \begin{otherlanguage}{czech}   ... \end{otherlanguage}
%%   \begin{otherlanguage}{slovak}  ... \end{otherlanguage}
%%
%% For non-Latin scripts, it may be necessary to load additional
%% fonts:
\usepackage{paratype}
\def\textrussian#1{{\usefont{T2A}{PTSerif-TLF}{m}{rm}#1}}
%%
%% The following section sets up the metadata of the thesis.
\thesissetup{
    date        = \the\year/\the\month/\the\day,
    university  = mu,
    faculty     = fi,
    type        = bc,
    department  = Katedra počítačových systémů a komunikací,
    author      = Denisa Maťátková,
    advisor     = {doc. Ing. RNDr. Barbora Bühnová, Ph.D.},
    title       = {Výuka programování na střední škole s využitím micro:bit},
    TeXtitle    = {Výuka programování na střední škole s využitím micro:bit},
    keywords    = {keyword1, keyword2, ...},
    TeXkeywords = {keyword1, keyword2, \ldots},
    abstract    = {%
      This is the abstract of my thesis, which can
    },
    thanks      = {%
      Děkuji.
    },
    bib         = example.bib,
    %% Remove the following line to use the JVS 2018 faculty logo.
    facultyLogo = fithesis-fi,
}
\usepackage{makeidx}      %% The `makeidx` package contains
\makeindex                %% helper commands for index typesetting.
\usepackage[acronym]{glossaries}          %% The `glossaries` package
\renewcommand*\glspostdescription{\hfill} %% contains helper commands
\loadglsentries{example-terms-abbrs.tex}  %% for typesetting glossaries
\makenoidxglossaries                      %% and lists of abbreviations.
%% These additional packages are used within the document:
\usepackage{paralist} %% Compact list environments
\usepackage{amsmath}  %% Mathematics
\usepackage{amsthm}
\usepackage{amsfonts}
\usepackage{url}      %% Hyperlinks
\usepackage{markdown} %% Lightweight markup
\usepackage{listings} %% Source code highlighting
\lstset{
  basicstyle      = \ttfamily,
  identifierstyle = \color{black},
  keywordstyle    = \color{blue},
  keywordstyle    = {[2]\color{cyan}},
  keywordstyle    = {[3]\color{olive}},
  stringstyle     = \color{teal},
  commentstyle    = \itshape\color{magenta},
  breaklines      = true,
}
\usepackage{floatrow} %% Putting captions above tables
\floatsetup[table]{capposition=top}
\usepackage[babel]{csquotes} %% Context-sensitive quotation marks

\begin{document}
%% Uncomment the following lines (by removing the %% at the beginning)
%% and to print out List of Abbreviations and/or Glossary in your
%% document. Titles for these tables can be changed by replacing the
%% titles `Abbreviations` and `Glossary`, respectively.
%% \clearpage
%% \printnoidxglossary[title={Abbreviations}, type=\acronymtype]
%% \printnoidxglossary[title={Glossary}]

%% The \chapter* command can be used to produce unnumbered chapters:
\chapter*{Úvod}
%% Unlike \chapter, \chapter* does not update the headings and does not
%% enter the chapter to the table of contents. I we want correct
%% headings and a table of contents entry, we must add them manually:
\markright{\textsc{Úvod}}
\addcontentsline{toc}{chapter}{Úvod}

Říká se, že závěrečné práce jsou \enquote{vyvrcholením studia}
a tak jsem se rozhodl jednu také napsat. Pokud vše půjde podle
plánu, odnesu si na konci semestru diplom. Držte mi palce!

\chapter{Kurikulární dokumenty a strategie vzdělávací politiky v oblasti informatiky}

\section{Vzdělávací politika}

\section{Kurikulární dokumenty}
\subsection{RVP G}
\subsection{ŠVP G}

\section{Výuka programování na SŠ}
    \subsection{Lego}
    \subsection{Arduino}

\chapter{Představení prostředků}
Robotické stavebnice jsou didaktickým prostředkem, který umožňuje na praktických příkladech pochopit principy programování. Současná práce s elektronikou i fyzickou stavebnicí podporuje rozvoj kreativity a logického myšlení. Robotické pomůcky u žáků velmi přirozeným způsobem rozvíjí schopnost algoritmizace, jako jedné ze základních složek informatického myšlení. Většina robotických stavebnic je programovatelná, jak pomocí vizuálních programovacích jazyků, tak i pomocí běžných jazyků, jako Python nebo C. Díky tomu jsou vhodné pro využití ve všech stupních vzdělávání.

\section{Robotické stavebnice}
Na trhu je velké množství pomůcek určených pro výuku programování, v následujících odstavcích bude představena jedna z nich a její hlavní výhody a nevýhody. 

\subsection{BBC Micro:bit}
Micro:bit je programovatelný mikropočítač, vytvořený s podporou BBC za účelem výuky programování a principů fungování počítačů. Výhodou zařízení BBC Micro:bit je, jeho finanční dostupnost a velikost. Micro:bit deska má velikost pouze 4 x 5 cma i přesto obsahuje mnoho funkcionalit. Existují tři základní možnosti programování micro:bitu a to prostřednictvím blokového programování, JavaScriptu a MicroPythonu. K dispozici jsou také dva oficiální textové editory, Microsoft MakeCode2 a MicroPython3.

Výhodou využití robotické stavebnice v porovnání s výukou pouze zmíněného Pythonu nebo JavaScriptu je zde možnost pochopit, princip fungování počítače komplexněji v kontextu hardwaru a nejrůznějších zařízeních založených na senzorech. Na rozdíl od dříve/později popisovaných zařízení, jako je Arduino či Raspberry Pi byl micro:bit navržen a vytvořen primárně za účelem vzdělávání a odpadá tak například spouštění plnohodnotného operačního systému a je možné více pozornosti věnovat informatickému myšlení.

\subsection{Nezha Kit}
Nezha Inventors Kit je robotická stavebnice navržená pro BBC micro:bit a je kompatibilní s první i druhou verzí. Tato sada pro vynálezce obsahuje několik senzorů PlanetX, díky nimž je možné se sadou vytvořit několik desítek různých projektů. Základ setu tvoří modul pro umístění micro:bitu. 

Pro propojení jednotlivých modulů jsou použity vodiče s konektory RJ11, které se používají například pro ethernetové připojení. Stačí zacvaknout a senzory jsou propojené s modulem a tedy i s micro:bitem. Propojení je snadné a spolehlivé. Další výhodou je kompatibilita s Nezha kitu se stavebnicí lego.

\section{Jazyky a prostředí}
Platforma micro:bit se díky možnosti blokového programování,, které je velmi populární pro svou jednoduchost a uživatelskou přívětivost, hodí pro úplné začátečníky. Zároveň se však neomezuje jen na bloky a umožňuje jednoduše přejít na Python JavaScript a případně i jiné jazyky.

\subsection{Blokové programování}
Nejvhodnějším způsobem, jak začít s programováním již u žáků základní školy, je blokové programování. To poskytuje úvod do strukturovaného programování prostřednictvím barevných bloků, které se přetahují a spojují to sekvencí, ty pak tvoří výsledný program.

Pro práci s blokovými jazyky je třeba jen, aby žáci uměli číst a pracovat s myší. Žáci se nemusí tolik trápit technickým řešením a můžou energii věnovat otázce, jak vyřešit problém, což jim umožní si rychleji osvojit algoritmické myšlení. Samotné skládání bloků připomíná puzzle, bloky do sebe zapadnou jen tehdy, není-li porušena syntaxe. Žáci tedy hned ví, že je něco špatně. Z hlediska výuky algoritmizace je výhoda, že se tím žáci zpočátku nemusí zabývat.

\subsubsection{Proměnné a datové typy}
Protože blokové programování využívá stejné struktury (např. cykly, podmínky, proměnné, atd.), jako běžné programovací jazyky, je pro žáky snazší na tomto pochopit princip fungování jednotlivých příkazů. 

Blokové jazyky obvykle však nepracují s typováním, a proto, to při přechodu na některý ze silně typovaných jazyků může z počátku dělat problémy. Bývá deklarován jen číselný datový typ, který se využívá k uchovávání počtu nějakých jednotek a následné kontrole např. za účelem ukončení programu. Deklarace proměnné probíhá pouze tak, že se vytvoří v sekci proměnné, tam se přiřadí pojmenuje a dále se s ní pracuje jako s libovolným jiným blokem.

\subsubsection{Úvod do programování - blokové jazyky}
Na základních školách se žáci pravděpodobně již setkali s některým blokovým jazykem.

Scratch je velmi oblíbený jazyk, který se dodává spolu s IDE (Integrated Development Envirnoment), které je multiplatformní. IDE je dostupné z desktopové instalace pro Windows, Mac OS a Linux, i jako webová verze, která se po zaregistrování a přihlášení se do účtu chová jako hub pro uchovávání a sdílení projektů. Scratch vznikl v roce 2005 na MIT jako projekt Mitchela Resnicka. A je poměrně hojně využívaná ve vzdělávání nebo výzkumech.

Dalším zástupcem je Robotanik, robot zahradník. Narozdíl od Scratche nezahrnuje cykly a větvení.

\subsection{Python}
Přestože v plánu úvodní hodiny budou pro demonstraci využity bloky, všechny ostatní jsou tvořené pro práci v Pythonu. Žáci by se dle RVP měli na základní škole setkat s algoritmizací, k čemuž jsou, jak již bylo zmíněno, právě blokové jazyky vhodné a často využívané. Ukázka, jak daný kód vypadá v blocích, které již znají, značně ulehčí přechod k Pythonu.

Python je vysokoúrovňový, interpretovaný programovací jazyk, který nabízí podporu pro různá programovací paradigmata. V případě microbitů si vystačíme s imperativním. Je dynamicky typovaný a tedy by žáci po přechodu z bloků nemuseli být zásadní problém. Syntaxe je založena na oddělování kódu pomocí bílých znaků, které oddělují jednotlivé bloky a přispívají k dobré čitelnosti. Zároveň tento způsob zápisu do jisté míry připomíná práci s bloky, které zapadají do sebe a tvoří podobnou strukturu.

Python je zvolen z mnoha důvodů. Jedním z nich je jednoduchá syntaxe. Například v porovnání s jazyky Java nebo C klade Python důraz na čitelnost a jednoduchost kódu. Stručná syntaxe jazyka umožňuje vyjadřovat mnohé koncepty v nižším počtu řádků kódu a udržovat tak v kódu přehlednost. I proto stále více výzkumníků tvrdí, že Python by se měl vyučovat jako první programovací jazyk. Ačkoli se pro výuku programování stále využívají i jiné jazyky začínající žáci v nich obvykle čelí problémům. V Pythonu se žáci z počátku nemusí zabývat třídami, metodami a jinými složitějšími konstrukty. První programovací jazyk by dle \textcite{Lo15} měl mít jednoduché vstupní/výstupní příkazy, snadno čitelnou a konzistentní syntaxi a být nezávislý \parencite{Lo15}.

\subsection{IDE}

\subsubsection{MicroPython}

\subsubsection{MakeCode Python}


\chapter{Metodické zpracování úloh}
The following chapter is Chapter and starts at page.


\chapter{Závěr}
The serified roman font is used for the main body of the text.


\printbibliography[heading=bibintoc] %% Print the bibliography.

\makeatletter\thesis@blocks@clear\makeatother
\phantomsection %% Print the index and insert it into the
\addcontentsline{toc}{chapter}{\indexname} %% table of contents.
\printindex

\appendix %% Start the appendices.


\chapter{Přílohy}
Here you can insert the appendices of your thesis.

\end{document}
