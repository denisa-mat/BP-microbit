%%%%%%%%%%%%%%%%%%%%%%%%%%%%%%%%%%%%%%%%%%%%%%%%%%%%%%%%%%%%%%%%%%%%
%% I, the copyright holder of this work, release this work into the
%% public domain. This applies worldwide. In some countries this may
%% not be legally possible; if so: I grant anyone the right to use
%% this work for any purpose, without any conditions, unless such
%% conditions are required by law.
%%%%%%%%%%%%%%%%%%%%%%%%%%%%%%%%%%%%%%%%%%%%%%%%%%%%%%%%%%%%%%%%%%%%

\documentclass[
  digital,     %% The `digital` option enables the default options for the
               %% digital version of a document. Replace with `printed`
               %% to enable the default options for the printed version
               %% of a document.
%%  color,       %% Uncomment these lines (by removing the %% at the
%%               %% beginning) to use color in the printed version of your
%%               %% document
  oneside,     %% The `oneside` option enables one-sided typesetting,
               %% which is preferred if you are only going to submit a
               %% digital version of your thesis. Replace with `twoside`
               %% for double-sided typesetting if you are planning to
               %% also print your thesis. For double-sided typesetting,
               %% use at least 120 g/m² paper to prevent show-through.
  nosansbold,  %% The `nosansbold` option prevents the use of the
               %% sans-serif type face for bold text. Replace with
               %% `sansbold` to use sans-serif type face for bold text.
  nocolorbold, %% The `nocolorbold` option disables the usage of the
               %% blue color for bold text, instead using black. Replace
               %% with `colorbold` to use blue for bold text.
  lof,         %% The `lof` option prints the List of Figures. Replace
               %% with `nolof` to hide the List of Figures.
  lot,         %% The `lot` option prints the List of Tables. Replace
               %% with `nolot` to hide the List of Tables.
]{fithesis4}
%% The following section sets up the locales used in the thesis.
\usepackage[resetfonts]{cmap} %% We need to load the T2A font encoding
\usepackage[T1,T2A]{fontenc}  %% to use the Cyrillic fonts with Russian texts.
\usepackage[
  main=czech, %% By using `czech` or `slovak` as the main locale
                %% instead of `english`, you can typeset the thesis
                %% in either Czech or Slovak, respectively.
  english, german, russian, czech, slovak %% The additional keys allow
]{babel}        %% foreign texts to be typeset as follows:
%%
%%   \begin{otherlanguage}{german}  ... \end{otherlanguage}
%%   \begin{otherlanguage}{russian} ... \end{otherlanguage}
%%   \begin{otherlanguage}{czech}   ... \end{otherlanguage}
%%   \begin{otherlanguage}{slovak}  ... \end{otherlanguage}
%%
%% For non-Latin scripts, it may be necessary to load additional
%% fonts:
\usepackage{paratype}
\def\textrussian#1{{\usefont{T2A}{PTSerif-TLF}{m}{rm}#1}}
%%
%% The following section sets up the metadata of the thesis.
\thesissetup{
    date        = \the\year/\the\month/\the\day,
    university  = mu,
    faculty     = fi,
    type        = bc,
    department  = Department of XXXXXX,
    author      = Denisa Maťátková,
    advisor     = {doc. Ing. RNDr. Barbora Bühnová, Ph.D.},
    title       = {Výuka programování a algoritmizace na střední škole s využitím jednodeskového počítače Micro:bit},
    TeXtitle    = {Výuka programování a algoritmizace na střední škole s využitím jednodeskového počítače},
    keywords    = {keyword1, keyword2, ...},
    TeXkeywords = {keyword1, keyword2, \ldots},
    abstract    = {%
      This is the abstract of my thesis, which can
    },
    thanks      = {%
      Děkuji.
    },
    bib         = example.bib,
    %% Remove the following line to use the JVS 2018 faculty logo.
    facultyLogo = fithesis-fi,
}
\usepackage{makeidx}      %% The `makeidx` package contains
\makeindex                %% helper commands for index typesetting.
\usepackage[acronym]{glossaries}          %% The `glossaries` package
\renewcommand*\glspostdescription{\hfill} %% contains helper commands
\loadglsentries{example-terms-abbrs.tex}  %% for typesetting glossaries
\makenoidxglossaries                      %% and lists of abbreviations.
%% These additional packages are used within the document:
\usepackage{paralist} %% Compact list environments
\usepackage{amsmath}  %% Mathematics
\usepackage{amsthm}
\usepackage{amsfonts}
\usepackage{url}      %% Hyperlinks
\usepackage{markdown} %% Lightweight markup
\usepackage{listings} %% Source code highlighting
\lstset{
  basicstyle      = \ttfamily,
  identifierstyle = \color{black},
  keywordstyle    = \color{blue},
  keywordstyle    = {[2]\color{cyan}},
  keywordstyle    = {[3]\color{olive}},
  stringstyle     = \color{teal},
  commentstyle    = \itshape\color{magenta},
  breaklines      = true,
}
\usepackage{floatrow} %% Putting captions above tables
\floatsetup[table]{capposition=top}
\usepackage[babel]{csquotes} %% Context-sensitive quotation marks

\begin{document}
%% Uncomment the following lines (by removing the %% at the beginning)
%% and to print out List of Abbreviations and/or Glossary in your
%% document. Titles for these tables can be changed by replacing the
%% titles `Abbreviations` and `Glossary`, respectively.
%% \clearpage
%% \printnoidxglossary[title={Abbreviations}, type=\acronymtype]
%% \printnoidxglossary[title={Glossary}]

%% The \chapter* command can be used to produce unnumbered chapters:
\chapter*{Úvod}
%% Unlike \chapter, \chapter* does not update the headings and does not
%% enter the chapter to the table of contents. I we want correct
%% headings and a table of contents entry, we must add them manually:
\markright{\textsc{Úvod}}
\addcontentsline{toc}{chapter}{Úvod}

Říká se, že závěrečné práce jsou \enquote{vyvrcholením studia}
a tak jsem se rozhodl jednu také napsat. Pokud vše půjde podle
plánu, odnesu si na konci semestru diplom. Držte mi palce!

\begin{otherlanguage}{czech}
Říká se, že závěrečné práce jsou \enquote{vyvrcholením studia}
a tak jsem se rozhodl jednu také napsat. Pokud vše půjde podle
plánu, odnesu si na konci semestru diplom. Držte mi palce!
\end{otherlanguage}

\chapter{Kapitola jedna}
\shorthandoff{-}
\begin{markdown*}{%
  hybrid,
  definitionLists,
  footnotes,
  inlineFootnotes,
  hashEnumerators,
  fencedCode,
  citations,
  citationNbsps,
  pipeTables,
  tableCaptions,
}

If you decide that \LaTeX{} is too wordy for some parts of your
document, there are that allow you to use more lightweight markup next
to it.

\end{markdown*}
\shorthandon{-}

\chapter{Kapitola dva}
\section{nadpis}
\subsection{podnadpis}
\subsubsection{podpodnadpis}
\paragraph{Paragraphs and}
\subparagraph{subparagraphs are available as well.}
Inside the text, you can also use unnumbered lists,

\chapter{Kapitola tři}

The logo of \gls{MU} is shown in Figure and
Figure at pages and. The weather forecast is shown in Table at page. The following
chapter is Chapter and starts at page.  Items are starred in the following list:


\chapter{Kapitola čtyři}
\LaTeX{} defines the automatically numbered \texttt{equation} environment:

For the complete list of environments and commands, consult the
\textsf{amsmath} package manual.

\chapter{Kapitola pět}
The serified roman font is used for the main body of the text.
\textit{Italics are typically used to denote emphasis or
quotations.} \texttt{The teletype font is typically used for source
code listings.} The \textbf{bold}, \textsc{small-caps} and
\textsf{sans-serif} variants of the base roman font can be used to
denote specific types of information.


\chapter{Inserting the bibliography}
This is just dummy text
\parencite{borgman03} lightly sprinkled with citations
\parencite[p.~123]{greenberg98}. Several sources can be cited at
once: \cite{borgman03,greenberg98,thanh01}.
\citetitle{greenberg98} was written by \citeauthor{greenberg98} in
\citeyear{greenberg98}. We can also produce \textcite{greenberg98}%
\ or %% Let us define a compound command:
\def\citeauthoryear#1{(\textcite{#1},~\citeyear{#1})}%
\citeauthoryear{greenberg98}%
. The full bibliographic citation is:
\emph{\fullcite{greenberg98}}.

  \printbibliography[heading=bibintoc] %% Print the bibliography.

  \makeatletter\thesis@blocks@clear\makeatother
  \phantomsection %% Print the index and insert it into the
  \addcontentsline{toc}{chapter}{\indexname} %% table of contents.
  \printindex

\appendix %% Start the appendices.


\chapter{Přílohy}
Here you can insert the appendices of your thesis.

\end{document}
